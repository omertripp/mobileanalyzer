\section{Advanced Features}

Having described the core algorithm, we now highlight challenges that arise in practice and advanced features that we have implemented in response.

\subsection{Indirect Access to Extra Parameters}\label{Se:indirectExtra}

Monitoring accesses to extra parameters is the core means for \Tool\ to expand its coverage of the IAC attack surface. In \secref{corealg}, we illustrate the core functionality of tracking direct access to extra parameters via {\tt Intent} methods like {\tt getStringExtra($\ldots$)}. There is, however, an alternate way of accessing custom parameters, which we demonstrate in \figref{techExample} using the commented-out statements with label {\tt A1}.

Instead of accessing custom parameters via the {\tt Intent} object, the {\tt Bundle} associated with the {\tt Intent} is retrieved via a call to {\tt Intent.getBundle()}. From this point on, custom parameters can be accessed through the {\tt Bundle} object using APIs like {\tt Bundle.getString($\ldots$)}. We refer to this method of accessing custom parameters as \emph{indirect}, because a given {\tt Bundle} may or may not be linked to the {\tt Intent} sent by \Tool.

The first challenge is to establish the binding between the {\tt Intent} sent by \Tool\ and its respective {\tt Bundle}. This is done by installing a monitor inside the {\tt Intent.getBundle()} method. The monitoring code (i) checks whether the receiving {\tt Intent} is indeed the one sent by \Tool\ (by inspecting the {\tt uri} and extra values), and if so, then (ii) the {\tt Bundle} is marked as \emph{relevant}. Furthermore, when a {\tt Bundle} is cloned (which is done via a copy constructor), then the clone is marked as relevant as well. This is achieved by placing monitoring code within the copy constructor.

Based on this marking technique, \Tool\ is able to distinguish between accesses to relevant versus irrelevant {\tt Bundle} objects. Only custom parameters queried over a relevant {\tt Bundle} object are recorded for future investigation. Other custom-parameter accesses are ignored.

\subsection{Coverage Expansion via Boolean Extra Parameters}\label{Se:coverBools}

While string values carried by the {\tt Intent} present an opportunity for an attack, boolean values are not candidate attack targets, and so the naive approach is to simply ignore such input points. In practice, boolean flags are typically used by the code that handles the {\tt Intent} to decide how to process the incoming data. That is, boolean extra parameters often manifest in (or as) path conditions.

As an example, we refer to the commented-out conditional expression in \figref{techExample}, annotated with the {\tt A2} label. It contains four conditional expressions, which refer to boolean extra {\tt ``b1''} in the outer condition and {\tt ``b2''} and {\tt ``b3''} separately and then conjoined in the nested condition. By default all three {\tt execSQL($\ldots$)} sink calls governed by these conditions will be missed by \Tool, as the default value assigned to {\tt ``b1''}, {\tt ``b2''} and {\tt ``b3''} (if they are not set on the {\tt Intent}) is {\tt false}. 

We have verified experimentally, over real-world apps, that coverage loss due to ignoring boolean extras is noticeable. (See \secref{boolExtras}.) On the other hand, 
brute-force enumeration of all possible boolean combinations has exponential complexity, and is therefore undesirable for performance reasons. (Again, see \secref{boolExtras}.) 

\Tool\ aims for a practical tradeoff between performance and coverage. This is achieved in light of two observations:
\begin{itemize}
	\item Domination: We say that boolean extra $b$ \emph{dominates} another boolean extra $b'$ if retrieval of $b'$ from the {\tt Intent} is attempted when $b$ is assigned one of the truth values but not its negation. In the example in \figref{techExample}, {\tt ``b1''} dominates both {\tt ``b2''} and {\tt ``b3''}. The runtime monitoring code installed within the {\tt getBooleanExtra($\ldots$)} method reveals to \Tool\ that access to {\tt ``b2''} and {\tt ``b3''} is made only when {\tt ``b1''} is set to {\tt true}.
	\item Independence: In practice, distinct boolean extras rarely flow into the same path condition. We have validated this experimentally. (See \secref{boolExtras}.) The upshot is that it is more effective to toggle one boolean extra at a time (linear complexity) rather than exhausting all possible value combinations (exponential complexity). As a limitation, \Tool\ would miss cases such as {\tt ``b2''} $\&\&$ {\tt ``b3''} in \figref{techExample}, which require simultaneous toggling of both {\tt ``b2''} and {\tt ``b3''}.
\end{itemize}

The combination of domination and independence leads to the following efficient algorithm for searching through assignments to discovered boolean extras: 
\begin{itemize}
	\item Domination: For an {\tt Intent} that exposes domination of boolean extra $b$ by  $b'$, only $b$ is toggled.
	\item Independence: For an {\tt Intent} that exposes $n$ boolean extras $b_1 \ldots b_n$, where none of the booleans are pairwise related via domination, each of the booleans $b_i$ is negated in turn while retaining the previous value of the other booleans.
\end{itemize}
Independence reduces complexity from exponential to linear, and domination further reduces the number of assignments to be explored, thereby yielding an efficient exploration procedure.