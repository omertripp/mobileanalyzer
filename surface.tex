\section{The IAC Attack Surface}\label{Se:surface}

In this section, we provide technical background on Android IAC, and then describe a catalog of security vulnerabilities that are all exploitable via the IAC surface. These are based on publicly available vulnerability reports \cite{XAS-CVE} as well as on one of the authors' field experience in performing manual security audits.

\subsection{Background: The Android Architecture}\label{Se:andarchitecture}

An Android app consists of components of the following types:
{\tt Activity}, {\tt Fragment}, {\tt Service},
{\tt BroadcastReceiver} and {\tt ContentProvider}. 
%
All the components comprising an app (except dynamically registered {\tt BroadcastReceiver}s)
are declared in the app's manifest
file (\texttt{AndroidManifest.xml}).

An {\tt Activity}
defines a single UI screen, e.g. a browsing window or preferences dialog. A {\tt Fragment} is
a finer-grained UI container that is reusable within the same app. {\tt Service}s perform background tasks. Finally, the role of {\tt BroadcastReceiver}s is to intercept incoming messages from other application components and {\tt ContentProvider}s.
%
Another (potential) component in an Android app is \textit{native code}. Such code,
typically written in C or C++, interfaces with the Java code via the Java Native Interface (JNI).

Android apps are executed in a sandboxed environment
to protect both the system and the hosted applications from
malware \cite{EOM:SP09}. 
%
%The sandbox prevents malware from abusing system services,
%such as sending SMS messages without sufficient permissions. It also ensures data confidentiality.
%An app cannot access sensitive information held by any other app
%without sufficient privileges. For example, the Android browser
%stores sensitive information, such as cookies and the browsing cache and history. This data 
%cannot be accessed by third-party apps. 
%
The Android sandbox relies on, and augments, the Linux kernel's isolation
facilities.
%
%For instance, each Android app package is assigned a unique
%Linux user ID (UID), such that resources created by one app cannot be accessed by
%default by another app. An android app may request specific
%privileges (or permissions) during its installation. If these are granted by the
%user, then the app's capabilities are extended by relaxing security controls correspondingly.
%Permissions are defined in the application's manifest.
%
While sandboxing is a central security feature, it comes at the expense of interoperability. In many common situations, apps require the ability to interact. For example, the browser app should be capable of launching the Google Play app if the user points toward the Google Play website. 

To recover interoperability, Android provides high-level IAC mechanisms via the {\tt Binder} class, implemented as a driver in the Linux kernel. IAC is achieved via {\tt Message}s (for {\tt Service}s) and {\tt Intent}s (for {\tt Service}s as well as other components). {\tt Intent}s are messaging objects that contain both the payload and the target application component. Content is placed into the {\tt uri} (or {\tt data}) attribute. and can additionally be stored within an associated {\tt Bundle} in the form of extra parameters. 

{\tt Intent}s can either be \emph{implicit}, which means that the target is not specified, or \emph{explicit}, which means that a specific target is provided. 
%
{\tt Intent}s can be broadcast
to {\tt BroadcastReceiver}s, invoke {\tt Activity}s, or launch
a {\tt Service}. An application component can only be invoked by external parties, via an {\tt Intent}, if the manifest file permits that. The manifest also defines the permissions that the external party must possess.

\subsection{Attacking Apps via IAC}

We refer to an Android component as \emph{public} if (i) it is exported, either explicitly or implicitly, via {\tt Intent} filters; (ii) it requires neither signed nor system permissions;\footnote{
	Other permissions are either enabled implicitly or granted by the user at install time.
} or (iii) unsanitized data emanating from a public component flows into it.

Public components form a hole in the Android sandbox. They expose themselves to incoming data from malicious parties, which may lead to vulnerabilities if the data is not sanitized or validated. {\Frag}s are also potentially vulnerable, as they have access to incoming IAC data via their enclosing {\tt Activity} and through its initiating {\tt Intent}.

Malicious parties are both local and remote. Malware, highly prevalent in Android, can interact with the public IAC interfaces directly through explicit {\tt Intent}s without any special permissions. Another attack vector is \emph{drive-by exploitation}: Using the {\tt intent} URI scheme, malicious websites can drive the browser to generate {\tt Intent}s targeting the public IAC interface of an arbitrary app \cite{PHAB:ACSAC10,E:driveby,T:driveby}. These may be either implicit or explicit depending on the browser.
%\begin{figure}
%\includegraphics[scale=0.5]{AttackOutline}
%\caption{Attacking Activities via IAC\label{fig:Attacking-Activities-via-IPC}}
%\end{figure}

%To reason about the severity of IAC vulnerabilities, we introduce the notion of \emph{depth}, which is the number of user-triggered IAC calls required to invoke an application component. Exported {\Act}s (without permissions), for example, have depth 0. Non-exported {\Act}s that are automatically invoked by depth-0 {\Act}s
%are also of depth 0. Private components are all of depth $\infty$. Naturally, the most acute security vulnerabilities are those manifesting at depth 0.

\subsection{Vulnerabilities Due to IAC Channels}\label{Se:attackSurface}

Unsafe handling of incoming IAC data can result in different forms of attack. Below we list the main threats.

\cheapar{Cross-Application Scripting (XAS)} Similarly to Cross-Site Scripting (XSS) in the Web landscape, XAS arises when script content (mostly JavaScript code) is injected into the HTML UI of a hybrid mobile application. This enables different forms of attack, including (i) access to sensitive information, (ii) invocation of native code via JavaScript and (iii) UI defacing/rewriting to trigger phishing attacks.

A concrete interface point for XAS attacks is the {\tt WebView} class, which is used to render HTML content within a mobile app \cite{LHDWY:ACSAC11}. The principal method of {\tt WebView} is {\tt loadUrl($\ldots$)}. If a malicious app can control the current URL, then all the attacks listed above turn into potential threats. 
%We describe several concrete scenarios below.
%
To exploit an XAS vulnerability, an attacker can inject JavaScript code using either the {\tt javascript} URI scheme or the {\tt file} scheme. In Android versions up to Jelly Bean, the latter attack vector is relatively straightforward: The attacker creates a malicious HTML file, and directs the target {\tt WebView} object to load that file via an {\tt Intent}.
%
%Such code will run in the context of the currently loaded page, if such a page exists. Another possibility is to inject the payload using the {\tt file} scheme. In Android versions up to Jelly Bean, this attack vector is relatively straightforward: The attacker creates a malicious HTML file, and directs the target {\tt WebView} object to load that file via an {\tt Intent}.

\cheapar{{\tt Fragment} Injection} The static {\tt instantiate(Context ctx,String fname,Bundle args)} method of class {\tt Fragment} accepts as {\tt fname} the name of the {\tt Fragment} subclass to load reflectively. This can be leveraged by an attacker to load arbitrary code obtainable through the class loader of {\tt ctx}. 

A successful {\tt Fragment} injection attack can result in loading of an attacker-selected class into the context of the vulnerable app, which grants that class the same privileges and access rights as its host app.
Otherwise an exception is thrown, but prior to that, the class' static initializer and default constructor are executed, which creates another attack/privacy vector.

Another alternative is to load a {\tt Fragment} already defined by the application or Android/Java framework, but inject malicious initialization data into the {\tt Fragment}. {\tt Fragment}s that are normally loaded by private {\tt Activity}s are more likely to \emph{trust} rather than \emph{validate} their initialization arguments, which renders them more exploitable to {\tt Fragment} manipulation attacks.
%
%MARCO: I had already erased the following sentence because it is a repetition.  A few lines above we wrote: "First, the class' static initializer is run (if it has not run before). Second, the class' default constructor is executed followed by other lifecycle methods, such as {\tt onCreate($\ldots$)}."  So the following sentence is a repetition.  I am erasing it again.  Feel free to put it back though if you feel like it should stay.
%Note that in {\tt Fragment} manipulation, not only the static initializer and default constructor of the {\tt Fragment} subclass are invoked, but also other lifecycle methods like {\tt onCreate($\ldots$)}. 


%Even if no sensitive
%action is performed in automatically invoked methods, the attacker
%can be the device owner himself or a thief (thus he can control the
%UI and cause the loaded fragment to perform some action), and use
%this technique in order to attack system applications and bypass restrictions
%(see Section \ref{sec:Real-World-Example} for an example of attacking
%the Settings app).
%
%TODO: change diagram to a general attack.

%\begin{figure}[H]
%\includegraphics[scale=0.5]{Constructors}\caption{Exploitation by constructors\label{fig:Exploitation-by-constructors-1}}
%\end{figure}


%\begin{figure}[H]
%\includegraphics[scale=0.5]{Fragments}
%\caption{Exploitation by Fragments\label{fig:Exploitation-by-Fragments-1}}
%\end{figure}

\cheapar{Client-side SQL Injection} Android provides support for SQLite databases. Applications can create
and manipulate a private database via a string handle. Once created, the database
is stored in a subdirectory of the directory allocated for the app
with the chosen handle as the file name. 

The Android SQL
support exposes Android applications to the full spectrum of SQL injection (SQLi) vulnerabilities.
Insecure invocation of a database operation (for example, \texttt{SQLiteDatabase.execSQL($\ldots$)}), can result in an SQLi vulnerability that may have
implications on both the integrity and confidentiality of the vulnerable
app.

\cheapar{File Manipulation} Android
app packages are each assigned a unique Linux UID. A direct consequence
is that files created by an app package cannot be accessed,
let alone tampered with, by another app. Though an attacker cannot directly manipulate a file, if (s)he has control over a parameter specifying the path to a given
 file used by the vulnerable app, then (s)he can subvert
the integrity and/or confidentiality of the file.

\cheapar{Native Memory Corruptions} As mentioned in \secref{andarchitecture}, Android apps may include
native code written in C/C++. Such code is subject
to classic memory corruption issues like buffer overflow, 
format-string violations and dangling pointers. Such weaknesses can be leveraged to execute arbitrary
code in the context of the vulnerable app.


\cheapar{Unhandled Exceptions (Denial of Service)} Programming errors that trigger unchecked exceptions (like null dereference)
will usually cause the target app to crash if the exception
is not caught. This presents an opportunity for denial-of-service (DoS) attacks, and can more generally drive the application into an unexpected state.