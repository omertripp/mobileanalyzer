\section{Conclusion and Future Work}\label{Se:conclusion}

We have investigated the integrity implications of unsafe handling of incoming IAC traffic in Android.
While this is a natural attack surface, we were surprised to learn that it has received little attention to date. We have assembled a catalog of eight different concrete attack vectors in this context, seven of which we confirmed to manifest in the wild: Out of 80 apps we examined, we found 52 to contain IAC vulnerabilities for a total of 155 distinct vulnerabilities.

As part of our study, to scale and simplify the detection of IAC integrity vulnerabilities, we have created \Tool, an automated dynamic testing tool designed to trigger and validate IAC vulnerabilities. \Tool\ performs glass-box analysis by monitoring the internal behavior of the target app on the fly via debug breakpoints (without intrusive and/or offline code transformations). Thanks to its code-level monitors, \Tool\ is able to track accesses to implicit input points (in the form of extra parameters), thereby achieving higher coverage in exploration (boolean extras) and testing (string extras). 

We report on experiments over \Tool, wherein \Tool\ was able to automatically detect 150 out of the 165 vulnerabilities in our benchmark suite with an average scanning time of 19 minutes per app. Our experiments further validate the design choices underlying \Tool. \Tool\ is currently featured in a commercial security-assessment service.

In the future, we would like to extend \Tool\ to handle more aspects of Android IAC. Such aspects include (i) fuzzing of integral extra parameter, (ii) attacking other components beyond {\tt Activity}s and (iii) accounting for attack scenarios that cross component boundaries, such that the payload flows across multiple components (cf. \cite{RALB:ARES14}). In addition, we would like to investigate whether, and which, analogous forms of attack exist in other platforms (primarily iOS).


