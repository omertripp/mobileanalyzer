\section{Related Work}\label{Se:related}

We divide our discussion of related work into three categories: (i) research motivating testing for IAC integrity vulnerabilities, (ii) existing tools and solutions for detection of Android vulnerabilities and (iii) online prevention techniques. We discuss each in turn.

% http://ieeexplore.ieee.org/xpls/abs_all.jsp?arnumber=6263963

\paragraph{Motivation} Maji et al. conduct an experimental evaluation of the robustness of IAC in Android \cite{MAB:DSN12}. In their experiments, they perform fuzz testing with illegal {\tt Intent} inputs, e.g. with blank or random action or data fields, random extras, etc. Their results indicate that exceptional situations are often handled poorly. Even more severely, Maji et al. find that an unprivileged process can crash the Android runtime via IAC channels. 
%
In light of these findings, they draw attention to the security implications of unsatisfactory handling of IAC inputs, which provides a strong motivation for \Tool. 

%% http://ieeexplore.ieee.org/xpls/abs_all.jsp?arnumber=6228987
%
%The need for specialized testing for mobile apps
%
%% http://css.csail.mit.edu/6.858/2012/readings/android.pdf

Enck et al. survey security aspects of the Android architecture \cite{EOM:SP09}. They point to inherent risks of {\tt Intent}-based communication, such as unintentionally leaking information to explicitly listening attackers or receiving messages from unexpected senders if a public component is not assigned (sufficient) access permissions. The conclusion Enck et al. draw is that while Android provides comprehensive security features, these are nontrivial to use correctly. Indeed, many of the vulnerabilities discovered by \Tool\ stem from misuse or misunderstanding of Android security features. 

% http://delivery.acm.org/10.1145/2440000/2435379/p209-rastogi.pdf?ip=129.34.20.23&id=2435379&acc=ACTIVE%20SERVICE&key=90259622823F2C32%2E4A4E901FFFFFEB40%2E4D4702B0C3E38B35%2E4D4702B0C3E38B35&CFID=281389948&CFTOKEN=24789755&__acm__=1399997630_9dbc819e372867864ad3011ad7c79d7b

\paragraph{Detection} The AppsPlayground framework, developed by Rastogi et al. \cite{RCE:CODAPSY13}, performs dynamic security analysis of Android apps to detect malware as well as grayware apps.  AppsPlayground has a modular deign, enabling usage of different exploration techniques, including event triggering, fuzz testing and context-sensitive crawling. The choice of detection technique is also modular, permitting e.g. taint tracking, API monitoring and kernel-level monitoring. 
%
\Tool\ also performs dynamic monitoring, though unlike to AppsPlayground, \Tool's main functionality is to inject security payloads into the subject app. In our approach, the role of monitoring is not to only detect illegal behaviors but also to expand the testing surface.

% http://dl.acm.org/citation.cfm?id=2000018

The ComDroid tool, developed by Chin et al. \cite{CFGW:MOBISYS11}, detects IAC vulnerabilities statically. Specifically, ComDroid performs bounded interprocedural flow-sensitive tracking of {\tt Intent} objects. ComDroid is sensitive to {\tt Intent} properties, such as whether the {\tt Intent} contains extra parameters or defines an action. A warning is issued if an implicit {\tt Intent} is sent out with no or weak permission requirements. The specific warning is decided based on the {\tt Intent} state. As an example, if the {\tt Intent} carries extra data, then beyond the threats of theft and hijacking, there is also the threat of eavesdropping. 
%
While ComDroid focuses on outgoing IAC, \Tool\ tests for injection vulnerabilities. Furthermore, the granularity of the attack surface tested by \Tool\ is finer than ComDroid, which unlike \Tool\ is a static analysis tool.

\paragraph{Prevention} % http://dl.acm.org/citation.cfm?id=2381948
Kantola et al. \cite{KCHW:SPSM12} address IAC vulnerabilities due to (i) exposure of app-internal messages to third parties or (ii) receipt of external messages by internal components of the app. To reduce the threat of such vulnerabilities, Kantola et al. modify the Android heuristics to determine the eligible senders and recipients of messages. As an example, if an implicit {\tt Intent} can be sent to any component in the same app, then the modified heuristic would restrict the {\tt Intent} to intra-app communication.
%
\Tool, and IAC testing in general, complement this approach in highlighting where heuristics are needed and which heuristics to apply.

Another proposal, by Cozzette et al. \cite{CLMOABFKNR:IM13}, is to revise {\tt Intent} handling such that the system can only err on the safe side. Contrary to the current design, whereby a developer may inadvertently create opportunities for another app to inject a malicious {\tt Intent} or intercept outgoing {\tt Intent}s, Cozzette et al. propose a model that disables all such behaviors unless these are requested explicitly by the developer. Interestingly, \Tool\ remains relevant under the new model, since erroneous handling of incoming {\tt Intent}s remains possible even when the user is fully aware of the implications.

%% http://ieeexplore.ieee.org/xpls/abs_all.jsp?arnumber=6573084&tag=1
%
%The authors prevented these exploits by modifying Android's Intent handling behavior to err on the side of safety except where the developer seems to explicitly specify otherwise.
%
%% http://ieeexplore.ieee.org/xpls/abs_all.jsp?arnumber=6641043
%
%Their approach generates test cases to check whether components are vulnerable to attacks, sent through intents, that expose personal data.
%
%% http://selab.fbk.eu/ceccato/papers/2013/ast2013.pdf
%%
%This work spots mismatches between the intended behavior declared by an application and the observed functionalities implemented in its code.
%
% http://www.thinkmind.org/index.php?view=article&articleid=icsea_2012_23_10_10240
%
% Authors propose a security testing approach which aims to check whether Android applications are not vulnerable to malicious intents.